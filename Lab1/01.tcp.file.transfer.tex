\documentclass{article}
\usepackage[utf8]{inputenc}
\usepackage[english]{babel}
\usepackage{graphicx} %inserting images
\usepackage{float} %image placement
\usepackage{titling} %title placement
\usepackage{hyperref}  %clickable links
\usepackage{amsmath}
\usepackage{listings}
\usepackage{enumitem}
\usepackage{tikz}
\usetikzlibrary{shapes.geometric, arrows}
\tikzstyle{startstop} = [rectangle, rounded corners, minimum width=3cm, minimum height=1cm,text centered, draw=black, fill=purple!30]
\tikzstyle{process} = [rectangle, minimum width=3cm, minimum height=1cm, text centered, draw=black, fill=green!30]
\tikzstyle{ellipse} = [circle, minimum width=3cm, minimum height=1cm, text centered, draw=black, fill=blue!30]

\tikzstyle{arrow} = [thick,->,>=stealth]

\hypersetup{
    colorlinks=true,   % Enable colored links
    linkcolor=black,    % Color of internal links (sections, TOC, etc.)
    urlcolor=blue      % Color for URLs
}
\lstset{
    frame=single,             % Add a frame around the code
    numbers=left,             % Line numbers on the left
    numberstyle=\tiny,        % Line number style
    basicstyle=\ttfamily\footnotesize, % Code font
    keywordstyle=\color{blue}, % Keywords in blue
    stringstyle=\color{red},  % Strings in red
    commentstyle=\color{green!50!black}, % Comments in green
    backgroundcolor=\color{gray!10}, % Light gray background
    tabsize=4,                % Tab size
    showstringspaces=false,   % Don't show spaces in strings
    breaklines=true,          % Automatic line breaking
    captionpos=b,             % Caption below the code
    language=Python           % Language for syntax highlighting
}


\title{Report Labwork 1}
\author{
    Tran Minh Phuong - 22BI13366\\  
}
\date{November 2024}

\setlength{\droptitle}{-5em} % Adjust the vertical space before the title

\begin{document}

\begin{figure}[H]
    \textsc{\large \bfseries University of Science and Technology of Ha Noi}\\[0.5cm]
    \centering
    \includegraphics[width=0.8\linewidth]{usth.png}
    \label{fig:project-overview}

    \vspace{1cm}
 
    \vspace{1cm}

    \textsc{\Large \bfseries Distributed System}\\[1cm]
    
\end{figure}


% Manual title to avoid page break
\vspace{1cm}
\begin{center}
    \huge \textbf{\thetitle} \\[0.5cm]
    \Large \theauthor [0.5cm]
    \large \thedate
\end{center}

\vspace{4cm}
\newpage
\tableofcontents

\newpage
\section{Introduction}
The target of the first practical lab work is transferring file from a client side to a server side over TCP/IP in CLI(Command Line Interface). This assignment plays an important role in understanding how two networks communicate through TCP protocol.

\subsection{TCP and File Transfer over Sockets}
TCP-Transmission Control Protocol-guarantees reliable, ordered, and error-checked delivery of data. File transfer over sockets involves the following:
\begin{itemize}
    \item Establish a connection between the client and the server.
    \item Data transfer from the client and server.
    \item Properly handling connection closure and EOF (End of File).
\end{itemize}

\section{Protocol Design}
Below is the interaction diagram showing how the client and server communicate:

\FloatBarrier
\begin{figure}[H]
    \centering
    \begin{tikzpicture}[node distance=2cm]

    % Server nodes
    \node (start) [startstop] {Server};
    \node (bind) [process, below of=start] {Bind and Listen};
    \node (accept) [process, below of=bind] {Accept Connection};
    \node (receive) [process, below of=accept] {Receive File};
    \node (close) [startstop, below of=receive] {Close Connection};

    % Arrows for Server
    \draw [arrow] (start) -- (bind);
    \draw [arrow] (bind) -- (accept);
    \draw [arrow] (accept) -- (receive);
    \draw [arrow] (receive) -- (close);

    % Client nodes
    \node (start_client) [startstop, right of=start, xshift=6cm] {Client};
    \node (connect) [process, below of=start_client] {Connect to Server};
    \node (check) [process, below of=connect] {Check File in System};
    \node (send) [process, below of=check] {Send File};
    \node (close_client) [startstop, below of=send] {Close Connection};

    % Arrows for Client
    \draw [arrow] (start_client) -- (connect);
    \draw [arrow] (connect) -- (check);
    \draw [arrow] (check) -- (send);
    \draw [arrow] (send) -- (close_client);

    % Connection Arrows
    \draw [arrow] (connect.west) -- ++(-2.5,0) |- (accept.east);
    \draw [arrow] (send.west) -- ++(0,0) |- (receive.east);

    \end{tikzpicture}
    \caption{Client-Server Interaction for TCP File Transfer}
    \label{fig:protocol}
\end{figure}
\FloatBarrier


\subsection{Logic of the Protocol}
\begin{enumerate}
    \item The server is initiated by binding to an IP address and port, then waits for connections from clients.
    \item The client starts the process by initiating a connection to the server.
    \item Once connected, the server waits for a file, the client checks the file name in the system and sends the entered file to the server
    \item After the file is fully transferred, both the client and server close the connection properly.
\end{enumerate}

\section{System Organization}
\subsection{System Architecture}
The server and the client are 2 main components of the system. Their roles are described in the following:
\begin{itemize}
    \item \textbf{Server:} Handles incoming connections and receive file from clients.
    \item \textbf{Client:} Connects to the server and sends files.
\end{itemize}

\subsection{Architecture Diagram}
\begin{figure}[h]
    \centering
    \begin{tikzpicture}[node distance=2.5cm]

    % Nodes
    \node (server) [ellipse] {Server};
    \node (network) [ellipse, right of=server, xshift=4cm] {Network};
    \node (client) [ellipse, right of=network, xshift=4cm] {Client};

    % Arrows
    \draw [arrow] (server.east) -- node[anchor=south] {Accept Connection} (network.west);
    \draw [arrow] (network.east) -- node[anchor=south] {Request Connection} (client.west);
    \draw [arrow] (client.west) -- node[anchor=north] {Send Data} (network.east);
    \draw [arrow] (network.west) -- node[anchor=north] {Receive from Client} (server.east);

    \end{tikzpicture}
    \caption{System Architecture for TCP File Transfer}
    \label{fig:system}
\end{figure}

\section{Implementation}
\subsection{Key Code Snippets}
\subsubsection{Server Code}
The server code is responsible for listening to connections and sending files:
\begin{lstlisting}[language=Python, caption=Server Code]
import socket

HOST = input("Your server ip address?\n") #enter your host ip address
PORT = 2109  # port

def start_server():
    server_socket = socket.socket(socket.AF_INET, socket.SOCK_STREAM)
    server_socket.bind((HOST, PORT))
    server_socket.listen(1)  
    print(f"Server listening on {HOST}:{PORT}...")

    while True:
        conn, addr = server_socket.accept()
        print(f"Connection established with {addr}")

        
        filename = b""
        while True:
            byte = conn.recv(1)
            if byte == b'\n':  
                break
            filename += byte
        filename = filename.decode('utf-8')  
        print(f"Receiving file: {filename}")

        
        with open(filename, 'wb') as file:
            while True:
                data = conn.recv(1024)
                if not data:  
                    break
                file.write(data)

        print(f"File '{filename}' received and saved in the server folder.")
        conn.close()  
        print(f"Connection with {addr} closed.")

if __name__ == "__main__":
    start_server()

\end{lstlisting}

\subsubsection{Client Code}
The client code connects to the server and receives the file:
\begin{lstlisting}[language=Python, caption=Client Code]
import socket
import os

SERVER_IP = input("Server's IP address you want to connect?\n")  
PORT = 2109 

def send_file():
    try:
        client_socket = socket.socket(socket.AF_INET, socket.SOCK_STREAM)
        print(f"Attempting to connect to {SERVER_IP}:{PORT}...")
        client_socket.connect((SERVER_IP, PORT))
        print(f"Connected to server at {SERVER_IP}:{PORT}")

        while True:
            filename = input("Enter the full file name (with extension): ")
            if os.path.isfile(filename):  # Ensure the file exists
                break
            print("File not found. Please try again.")

        client_socket.send(filename.encode('utf-8') + b'\n')  # Send filename with newline delimiter

        with open(filename, 'rb') as file:
            print(f"Sending file '{filename}' to the server...")
            while chunk := file.read(1024):  # Read in chunks of 1KB
                client_socket.send(chunk)

        print(f"File '{filename}' sent successfully to the server.")
        client_socket.close()

    except ConnectionRefusedError:
        print("Connection failed: Ensure the server is running and reachable.")
    except socket.timeout:
        print("Connection timed out: Server is taking too long to respond.")
    except Exception as e:
        print(f"An unexpected error occurred: {e}")

if __name__ == "__main__":
    send_file()


\end{lstlisting}

\section{Lab work execution}
The lab work was executed by me. The server side was running at a PC using Window 10 and client side was executed using laptop running Ubuntu 24.04.
\begin{itemize}
    \item \textbf{Window:} Run the server file and receive testwindow file from Ubuntu.
    \item \textbf{Ubuntu:} Run the client file and send the testwindow file to server 
\end{itemize}

\begin{figure}[h!]
    \centering
    \includegraphics[width=0.8\textwidth]{server1.png}
    \caption{The server initiates connection and receive a file from the client}
    \label{fig:client}
\end{figure}

\begin{figure}[h!]
    \centering
    \includegraphics[width=0.8\textwidth]{server2.jpeg}
    \caption{The received file is the same as the file in client side}
    \label{fig:client}
\end{figure}



\section{Conclusion}
This labwork demonstrates the implementation of client and server connection over TCP/IP by Python which ensures correctness of the data content.

\end{document}